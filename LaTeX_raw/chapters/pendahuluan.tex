\chapter{PENDAHULUAN}
\section{Latar Belakang}
Berdasarkan hasil survei dari kominfo, penetrasi penggunaan internet di Indonesia tahun 2019 – 2020 telah mencapai angka 73,7\%. Dengan kata lain, pengguna internet di Indonesia diperkirakan telah mencapai angka 196,7 juta pengguna dari total keseluruhan penduduk di Indonesia yaitu 266.911.900 juta jiwa [1]. Dari data tersebut, terlihat jelas bahwa Indonesia telah berada di era baru yang mana internet serta teknologi menjadi hal yang lumrah bagi masyarakat awam.Melihat peluang ini, perusahaan-perusahaan di Indonesia mulai melakukan investasi besar besaran di bidang ini. Terbukti dengan jumlah startup pada tahun 2018 yang telah mencapai angka 992 perusahaan rintisan [2].
\par
Selain meningkatnya jumlah startup di Indonesia, peningkatan juga terlihat pada perubahan gaya hidup masyarakat Indonesia. Perubahan ini membuat masyarakat Indonesia yang terbiasa membeli barang secara konvensional, berubah menjadi pembelian serba online. Hal ini terlihat dari banyaknya berbagai layanan jasa toko online maupun transportasi online yang ada. 
\par
Melihat pola trend yang mengarah pada pembelian serba online ini, kami mengambil inisiatif untuk mengembangkan sebuah aplikasi website baru yang bergerak dalam bidang e-commerce, untuk menjual berbagai jenis desain pada platform kami. Aplikasi penjualan desain di Indonesia sendiri masih cukup jarang, dan kurang dilirik. Oleh karena itu, kami ingin mengembangkan sebuah sistem yang dapat menggiring pasar desain, melalui sistem dan fitur yang kami tawarkan di aplikasi ini.
\par
Meskipun platform e-commerce yang menjual desain belum terlalu dilirik, bidang ini sebenarnya sangatlah menjanjikan di masa depan. Mengingat teknologi 3D printing sudah mulai muncul di masyarakat awam, tentu saja ini akan meningkatkan tingkat kesuksesan platform ini di masa depan. Dengan adanya 3D printing, masyarakat hanya perlu membeli desain di platform ini dan melakukan print pada 3D printer tersebut. Sehingga tentu saja akan mengubah pangsa pasar di seluruh Indonesia.
\par
Untuk mewujudkan hal tersebut, kami mengembangkan sebuah aplikasi website bernama belidesain.com, yang merupakan sebuah platform e-commerce yang menjual berbagai jenis desain, antara lain. Desain interior, DKV, web design, tata busana, serta desain furniture. Dengan adanya platform desain yang serba ada ini, diharapkan masyarakat tergugah untuk mencari desain pada satu buah platform one for all, sehingga mereka tidak perlu mencari cari platform lain untuk membeli sebuah desain.
\par
Selain menjual desain, platform e-commerce ini juga berperan sebagai event organizer yang menyediakan lahan untuk menjual tiket expo perihal desain, yang nantinya akan membantu desainer yang ingin membuat pameran menjadi lebih mudah untuk berkarya. Serta di lain pihak, para penikmat desain dapat merasa nyaman sehingga tidak perlu mencari cari event mengenai expo bertajuk desain.
\par
Dan bukan hanya itu saja, selain merupakan platform penjual desain, aplikasi ini juga menyediakan desainer yang dapat dipanggil sesuai kebutuhan mereka. Sehingga client dapat meminta langsung mengenai custom desain yang mereka inginkan kepada desainer ini. Sehingga platform ini sangat menguntungkan bagi kedua pihak, baik client yang mencari seorang desainer, serta bagi pihak desainer yang mencari client.

\section{Rumusan Masalah}
Berdasarkan latar belakang yang telah dijelaskan di atas, maka rumusan masalah yang dapat diambil adalah sebagai berikut:
\begin{itemize}
	\item Bagaimana cara mempermudah masyarakat dalam menemukan desain?
	\item Bagaimana desainer dapat dengan mudah menjual desain mereka?
	\item Bagaimana desainer dapat terhubung dengan pelanggan mereka?
	\item Bagaimana membuat platform e-commerce berbasis website?
\end{itemize}

\section{Tujuan Penelitian}
Tujuan dari penelitian ini adalah untuk membuat platform e-commerce berbasis website yang berisi karya-karya dari para desainer interior dan eksterior.

\section{Manfaat Penelitian}
Manfaat yang diperoleh dari penelitian ini adalah sebagai berikut:
\begin{itemize}
	\item Bagi desainer: dapat memasarkan produk mereka dengan mudah.
	\item Bagi pengguna: mendapatkan informasi seputar desain dan desainer yang dibutuhkan.
	\item Bagi peneliti: dapat mengimplementasikan ilmu yang telah diperoleh di perkuliahan.
\end{itemize}

\section{Sistematika Penulisan}
Sistematika penulisan dari laporan kami:
\begin{itemize}
	\item Bab 1 : Pendahuluan \\ Menjelaskan mengenai latar belakang, rumusan masalah, tujuan penelitian, manfaat penelitian, dan sistematika penulisan.
	\item Bab 2 : Landasan Teori \\ Menjelaskan teori-teori yang berkaitan dengan permasalahan, dan tahapan pengembangan dalam pembuatan sistem.
	\item Bab 3 : Analisis \\ Berisikan use case diagram, identifikasi dari permasalahan dan identifikasi kebutuhan pengguna.
	\item Bab 4 : Berisikan tabel-tabel conceptual design, logical model dan physical model.
\end{itemize}