\chapter{LANDASAN TEORI}
\section{E-commerce}
Electronic Commerce (E-commerce) adalah aktivitas penjualan, pembelian atau pemasaran produk dan jasa melalui sarana elektronik seperti internet, televisi atau jaringan komputer lainnya. E-commerce melibatkan alat pembayaran, basis data, dan sistem pengiriman barang. E-commerce yang merupakan bagian dari e-bisnis menjadi proses bisnis yang dapat menghubungkan antara perusahaan, konsumen, dan masyarakat tanpa perlu bertatap muka secara langsung. Pembayaran dapat dilakukan secara online maupun offline. Dapat disimpulkan bahwa e-commerce memiliki karakteristik sebagai berikut:
\begin{itemize}
	\item Internet menjadi media utama dalam proses transaksi tersebut.
	\item Transaksi terjadi antara dua belah pihak.
	\item Adanya pertukaran informasi produk atau jasa
\end{itemize}
\par Berikut adalah beberapa jenis e-commerce yang sering diterapkan: \\\\
Business to Business (B2B)\\
\par Model bisnis yang terjadi antara mitra bisnis telah saling menjalin hubungan bisnis yang lama, sebab keduanya saling mendapatkan keuntungan dan adanya kepercayaan satu sama lain. \\\\
Consumer to Consumer (C2C)\\
\par Model bisnis yang melibatkan proses transaksi antar konsumen. Contohnya Tokopedia, Shopee, Blibli dan sejenisnya. Platform tersebut menjadi perantara prosesnya jual beli. \\\\
Consumer to Business (C2B)\\
\par Model bisnis yang terjadi dari konsumen ke perusahaan. Konsumen akan menawarkan produk atau jasa mereka kepada perusahaan yang membutuhkan. Contoh dari model bisnis ini adalah website freelancer.\\

\par Metode pembayaran pada e-commerce ada beberapa jenis seperti:	\\\\
Kartu Kredit atau Visa\\
\par Pembayaran jenis ini menjadi yang paling sering dilakukan dalam transaksi. Pemegang kartu hanya perlu mengisi data yang diperlukan, kemudian proses pembayaran akan otomatis terjadi. \\\\
E-Wallet\\
\par Pembayaran yang sekarang ini sangat populer pada transaksi online. Layanan yang diberikan masih terbatas pada beberapa pembayaran. Namun e-wallet mengalami perkembangan yang jauh lebih baik sekarang ini. Beberapa e-wallet yang sering dikenal yaitu OVO, Dana, dan Go-pay dari Gojek. \\\\
Cash on Delivery (COD)\\
\par Pembayaran yang dilakukan secara offline meskipun pembelian dilakukan secara online. Pembayaran berlangsung antara pembeli dengan penjual melalui perantara kurir. \\\\
Debit Visa\\
\par Pembayaran ini hampir mirip dengan kartu kredit. Perbedaannya adalah pemotongan biaya pada debit visa dilakukan pada rekening tabungan langsung. Contoh debit visa yaitu kartu dari Jenius.\\\\
Keuntungan dari penggunaan e-commerce adalah sebagai berikut:\\\\
Dapat menghemat waktu \\
\par Dengan pembelian secara online, tidak perlu melakukan perjalanan dari toko satu ke toko yang lain jika barang yang dicari tidak ada. Cukup mengecek website, memilih barang dan kemudian melakukan transaksi yang diperlukan. \\\\
Bisnis dapat dilakukan secara global \\
\par Jangkauan wilayah yang dijangkau tidak ada batasan. Transaksi dapat dijangkau oleh wilayah luar negeri tanpa harus pergi ke luar negeri langsung. Sehingga produk dapat dikenal hingga berbagai negara.\\\\
Modal yang diperlukan tidak terlalu besar\\
\par Modal yang diperlukan tidak sebanyak dengan membangun sebuah toko fisik. Sehingga akan menghemat biaya untuk membangun sebuah toko. Biaya pegawai yang dibutuhkan tidak akan sebanyak pegawai sebuah toko fisik. \\\\\
Dapat diakses kapanpun dan dimanapun \\
\par Hanya dengan bermodal kuota data, website dapat diakses dimanapun dan kapanpun. Transaksi pembelian dapat dilakukan dalam 24 jam dan apabila tidak sedang di rumah, transaksi tetap dapat dilakukan. \\\\
Persentase perkembangan bisnis lebih besar \\
\par Bisnis yang dapat dijangkau oleh siapapun dari berbagai penjuru menjadi faktor yang besar dalam berkembangnya bisnis. Selain itu, keuntungan yang didapatkan juga banyak karena biaya yang dikeluarkan tidak terlalu banyak.

\section{Konsep Database}
Jika kita memiliki banyak sekali buku, tentu saja kita memerlukan sebuah rak untuk menampung keseluruhan buku itu. Saat menyimpan buku dalam rak ini, terdapat dua solusi. Solusi pertama adalah menyimpan buku di sembarang tempat, dan kedua adalah menyimpan buku yang tersusun rapi dengan kode di rak tersebut dan lainnya. Kedua solusi ini memiliki kelebihan dan kekurangan masing masing. Jika kita mengikuti solusi pertama, maka keuntungan yang kita dapatkan adalah kemudahan dan kecepatan saat ingin menyimpan buku itu. Namun saat pencarian buku, kita akan kesulitan dalam mencarinya. Jika kita mengikuti solusi kedua, mungkin kita akan membutuhkan waktu yang lebih lama saat ingin menyimpan buku itu, namun tentunya kita memiliki kelebihan untuk lebih mudah mencari dan mengambil buku yang ada.
\par Itu adalah sebuah analogi dari betapa bermanfaatnya sebuah database. Kita dapat mengumpamakan buku tersebut sebagai data., dan rak buku kita ibaratkan sebagai database. Dengan adanya database kita dapat membuat kecepatan pemrosesan sebuah menjadi lebih cepat daripada tanpa database. Kita membutuhkan sebuah software DBMS untuk mengatur database. Software DBMS memiliki sistem penyimpanan sendiri yang lebih terstruktur sehingga saat melakukan pemanggilan data (query) akan berlangsung lebih cepat dibandingkan pemrosesan data yang disimpan pada file spreadsheet ataupun file lainnya.

\section{Definisi Database}
Database adalah himpunan kelompok data yang saling berhubungan yang diorganisasikan sedemikian rupa sehingga dapat dimanfaatkan kembali dengan cepat dan mudah. Prinsip utama dari database ini adalah pengaturan data, yang tujuan utamanya yaitu kemudahan dan kecepatan dalam pengambilan kembali data.

\section{Tujuan database}
Pemanfaatan database dilakukan untuk:
\begin{itemize}
	\item Kecepatan dan kemudahan (Speed).
	\item Efisiensi ruang penyimpanan (Space).
	\item Keakuratan (Accuracy).
	\item Ketersediaan (Availability).
	\item Kelengkapan (Completeness).
	\item Keamanan (Security).
	\item Pemakaian bersama (Shareability).
\end{itemize}

\section{Tahap pengembangan}
Penelitian ini bertujuan untuk mengembangan sebuah platform e-commerce berbasis website. Dalam pembuatan website diperlukan berbagai bahasa pemrograman seperti HTML, CSS, dan Javascript sebagai frontend clientside, serta PHP sebagai backend serverside dan MySQL sebagai database.
\par Frontend pada sebuah website adalah bagian yang langsung dilihat oleh user. Frontend bertanggung jawab agar user dapat berinteraksi pada website dengan nyaman. Kemampuan dasar yang dibutuhkan adalah bahasa pemrograman HTML, CSS, Javascript.
\par Hyper Text Markup Language (HTML) adalah pondasi dalam pembuatan website. Cascading Style Sheets (CSS) adalah bahasa yang mendukung HTML untuk membuat website yang estetik. CSS berfungsi untuk mengontrol tampilan HTML seperti warna, font, layout, dan style lainnya. Dan terakhir javascript yang berfungsi untuk membuat elemen interaktif seperti menu dinamis, dan animasi yang lebih kompleks sehingga website lebih menarik.
\par Backend bekerja dibalik layar dari sebuah website. User tidak dapat melihat atau berinteraksi langsung pada bagian ini. Backend lebih berfokus pada sistem dan fungsi daripada tampilan. Backend bertanggung jawab dalam semua hal yang berhubungan dengan server, seperti database, scripting dan arsitektur dari sebuah website.. 
\par PHP: Hypertext Preprocessor (PHP) berfungsi agar website menjadi lebih dinamis. PHP merupakan bahasa pemrograman server-side yang nantinya akan diproses di server. Selain itu, PHP bersifat open-source yaitu bebas memodifikasi danmengembangkan sesuai kebutuhan. PHP sering digunakan bersama dengan MySQL dalam membangun sebuah website. MySQL adalah sebuah sistem manajemen database yang dipakai dalam mengakses dan memproses data. 